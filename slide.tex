\documentclass[14pt]{beamer}

% font
\usepackage{fontspec}
\setmainfont[Ligatures=TeX]{IPAPGothic}
\usepackage{xeCJK}
\setCJKmainfont{IPAPGothic}
\newfontfamily{\liberation}{Liberation Sans}
\newfontfamily{\notosans}{Noto Sans CJK JP}

% graphic
\usepackage{graphics}
\usepackage{graphicx}
\usepackage{color}
\usepackage{xcolor}
\usepackage{colortbl}
\definecolor{mygray}{rgb}{0.1, 0.1, 0.1}

% tikz
\usepackage{tikz}
\usetikzlibrary{automata}
\usetikzlibrary{arrows}
\usetikzlibrary{arrows.meta}
\usetikzlibrary{positioning}
\usetikzlibrary{intersections, calc}
\usetikzlibrary{decorations}
\usetikzlibrary{decorations.markings}
\usetikzlibrary{decorations.pathreplacing,angles,quotes}
\usetikzlibrary{fit}
\usetikzlibrary{math}
\usetikzlibrary{shapes}
\usepackage{pgfplots}
\usepackage{bchart}

% href
\usepackage{hyperref}
\hypersetup{
	colorlinks=true,
	linkcolor=cyan,
	filecolor=cyan,
	urlcolor=cyan,
	pdfnewwindow=true}

% beamer
\usepackage{bxdpx-beamer}
\usetheme{Boadilla}
\setbeamertemplate{navigation symbols}{}
\setbeamercovered{transparent}
\setbeamertemplate{frametitle}{%
	\vspace{0.1em}
	\usebeamerfont{frametitle}\insertframetitle%
	\par
	\rule[0.5\baselineskip]{0.9\paperwidth}{0.4pt}%
	\vspace{-0.5em}}
\setbeamertemplate{footline}{
	\hfill
	\usebeamercolor[fg]{page number in head/foot}
	\usebeamerfont{page number in head/foot}
	{\small \insertframenumber}
	\kern1em\vskip5pt
}
\setbeamercolor{footline}{fg=black,bg=black}

% itemize
\usepackage{enumitem}
\setitemize{itemsep=0.3em}
\setlength\leftmargini{20pt}
\setlength\leftmarginii{20pt}
\setlength\leftmarginiii{20pt}
\setlength\leftmarginiv{20pt}
\setlist[itemize,1]{label=$\color{blue}\bullet$}
\setlist[itemize,2]{label=$\color{orange}\triangleright$}
\setlist[itemize,3]{label=$\color{gray}\bullet$}
\setlist[itemize,4]{label=$\color{red}\triangleright$}
\setlist[itemize,5]{label=$\color{gray}\bullet$}
\setlist[itemize,6]{label=$\color{red}\triangleright$}
\setlist[itemize,7]{label=$\color{yellow}\bullet$}
\setlist[itemize,8]{label=$\color{pink}\triangleright$}
\setlist[itemize,9]{label=$\color{black}\bullet$}

% math
\usepackage{amsmath,amssymb,amsthm}
\usepackage{bm}

% other
\usepackage{caption}
\usepackage{cancel}
\usepackage{epigraph}
\usepackage{fancybox}
\usepackage{here}
\usepackage{makecell}
\usepackage{setspace}
\usepackage{scrextend}
\usepackage{svg}
\usepackage{ulem}
\usepackage{multirow}

\begin{document}

\begin{frame}
	\begin{center}
		{\LARGE \color{cyan} トキポナのトークナイザを作ろう} \\
		\vspace{1em}
		{\scriptsize ミニマリズム言語のトークナイザならかんたんにできるよ}
	\end{center}
\end{frame}


\begin{frame}
	\frametitle{トークナイザを作りたい!}
	
	\begin{itemize}
		\item トークナイザ
			\begin{itemize}
				\item 機械でテキストを扱うためには,文章を単語や記号の単位に分割しなければならない
			\end{itemize}
		\item むずかしい!
		\item \alert{かんたん}な言語でなら作れる?
	\end{itemize}
	
\end{frame}

\begin{frame}
	\frametitle{トキポナとは}

	\begin{itemize}
		\item みんなが知ってるミニマリズム人工言語
			\begin{itemize}
				\item 語彙数: \alert{120語}
				\item 習得がかんたん たのしい!
				\item できる人が割と多い たのしい!
			\end{itemize}
		\item ごくわずかな知らない人のために
			\begin{itemize}
				\item \href{https://twitter.com/notolytos/status/1409484535151042568}{\small 初級トキポナ文法簡介}
				\item \href{https://www.youtube.com/watch?v=9C0YqTs4vB8}{\small 2分40秒で世界一簡単な言語を紹介して伝授する}
				\item \href{https://www.youtube.com/watch?v=wIFJfAhiPlE}{\small トキポナレッスン1「トキポナってなに」}
				\item \href{https://en.wikibooks.org/wiki/Updated\_jan\_Pije\%27s\_lessons}{\small jan Pije's lessons} 
					{\scriptsize (\href{https://github.com/stefichjo/toki-pona/blob/master/pije.md}{注})}
				\item \href{https://www.youtube.com/watch?v=2jRtYBaZGgQ}{\small 【日本語訳】toki pona li toki pona - トキポナソング}
			\end{itemize}
	\end{itemize}
\end{frame}

\begin{frame}
	\frametitle{トキポナのトークナイザを作りました}

	\begin{itemize}
		\item \href{https://github.com/nymwa/ilonimi}{github:nymwa/ilonimi}
			\begin{itemize}
				\item Python 3で書いた
				\item pipが使えれば使えます
			\end{itemize}
		\item この資料では,このトークナイザが何をやってるか説明します
	\end{itemize}
\end{frame}

\begin{frame}
	\frametitle{どのような処理を行うか?}
\end{frame}

\begin{frame}
	\frametitle{単語のリストを作る}
\end{frame}

\begin{frame}
	\frametitle{記号と単語を分離する}
\end{frame}

\begin{frame}
	\frametitle{固有名詞の判定をする}
\end{frame}

\begin{frame}
	\frametitle{固定名詞のオートマトン}
\end{frame}

\begin{frame}
	\frametitle{nmとnnを除外する規則}
\end{frame}

\end{document}

